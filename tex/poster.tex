\documentclass[a1paper,24pt]{tikzposter}

\title{Digitale Selbstlerneinheit}
\author{Cedric Geissmann}
\institute{PH FHNW - Wahl 1.2}

\colorlet{backgroundcolor}{white}
\colorlet{titlebgcolor}{white}
\colorlet{block}{white}

\begin{document}
\maketitle



\block{}{
    \def\arraystretch{1.5}
    \begin{center}
        
        \begin{tabular}{l l}
            \textbf{Verschlüsselungsverfahren}  & \\
            Fach      & Informatik \\
            Klasse    & 1. Gymnasium \\
            Lektionen & 1 (geht auch mehr) \\
        \end{tabular}
    \end{center}
    }
    \begin{columns}
        \column{0.5}
    \block{Allgemein}{
Die SuS haben bereits Informatik am Gymnasium gehabt und können für eine kurze Zeit konzentriert am Computer arbeiten.
Die SuS haben noch nichts über Verschlüsselung gelernt. Haben jedoch schonmal gehört wie Dateien funktionieren und haben ein grobes Verständnis davon wie Daten über das Netzwerk verschickt werden.

\hspace{1em} In dieser Lerneinheit sollen sich die SuS damit auseinander setzen, wie Daten verschlüsselt werden können, wenn man Sie über das Netzwerk verschickt.
}

\block{Lernziele}{
    Die SuS wissen, \dots
    \begin{itemize}
\item was einfache Verschlüsselungsverfahren sind.
\item weshalb Verschlüsselungsverfahren wichtig sind.
\item was ein Verfahren sicher macht.
\item wie man Schlüssel austauschen kann.
\item wie man Schlüssel für ein Verfahren verwenden kann.
    \end{itemize}

}

\block{Überfachliche Kompetenzen}{
In dieser Lerneinheit sollen vorallem die überfachlichen Kompetenzen \textbf{Selbstreflektion} und \textbf{Selbstüberprüfung} gefördert werden.
Diese gehören zu den metakognitiven Kompetenzen und sind für ein effizientes Lernen essentiell.
}

\column{0.5}

\block{Förderung der ÜfK}{
Als Werkzeug zur Selbsreflektion und Selbstüberprüfung wurde in der Lerneinheit ein Quiz eingebaut. Das Quiz soll es ermöglichen den Inhalt zu reflektieren, indem es das Verständnis kritisch hinterfragt. Das gelesene muss also nicht nur abgespeichert werden, sondern auch gleich angewendet.

\hspace{1em} Zur Selbstüberprüfung dient die Punktzahl die man im Quiz erreicht. Diese zeigt ob man ein Thema nochmals repetieren soll, oder weiter zum nächsten gehen kann.
}

\block{Taxonomie}{
Damit man sich vertiefter mit dem Material ausseinandersetzen kann, sind die Konzepte jeweils interaktiv animiert. Mit diesen Animationen kann man das Thema besser analysieren und das eigene Verständniss prüfen bzw. das gelernte anwenden.

\hspace{1em} Aus dem \textbf{Peerfeedback} ist zu entnehmen dass bei den Animationen Hinweise zum leiten des Lernenden benötigt werden.
}

\block{Herausforderungen}{
    Das erstellen der Fragen ist nicht trivial. Die Fragen so zu stellen dass tatsächlich reflektiert wird ist sehr schwierig. Hat auch mit der Motivationen der Lernenden zu tun. Das gleiche gilt für die Hinweise bei den Animationen.

    \hspace{1em} Denkbar wäre es erst weiter zu machen wenn eine gewisse Punktzahl in einem Quiz erreicht ist.
}

\end{columns}

\block{Digitale Selbstlerneinheit}{
    \begin{center}
    \includegraphics{qr_webseite}
    \end{center}
}


    
    
\end{document}
