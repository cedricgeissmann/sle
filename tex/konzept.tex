\documentclass{article}

\usepackage[utf8]{inputenc}
\usepackage[T1]{fontenc}
\usepackage[ngerman]{babel}
\usepackage{hyperref}

\author{Cedric Geissmann}
\title{%
Konzept für Selbstlerneinheit \\
\large Verschlüsselungsverfahren
}
\date{24. Mai 2023}

\begin{document}
\maketitle

\section*{Metakognition}

Selbstreflexion und Selbstüberwachung sind wichtige Konzepte um effizient selbst lernen zu können. Diese Konzepte sind aber nicht unbedingt gegeben, und müssen oft noch erlernt werden.

Gerade in einer digitalen Selbstlerneinheit kann man den Überblick sehr schnell verlieren wo man ist, oder durch eine kurze Recherche im Internet völlig ab von dem Thema kommen. Deshalb ist es wichtig, dass die digitalen Werkzeuge die wir verwenden, eine Möglichkeit zur Selbstreflexion und Selbstüberprüfung bieten.

In dieser Lerneinheit wird das ermöglicht, durch einen Überblick über das Thema, und am Ende von jedem Thema, gibt es ein Quiz, welches zur Selbstüberprüfung dient.

\section*{Lerneinheit}

In dieser Lerneinheit sollen sich die SuS mit Verschlüsselungsverfahren beschäftigen. Zu beginn kommen sehr einfache Verfahren die man auch mit Stift und Papier durchführen kann. Diese Verfahren dienen dazu dass die SuS sich aktiv Gedanken machen sollen, was denn Verschlüsselung eigentlich ist. Dabei sollen Sie merken dass es darum geht einen Text unleserlich zu machen, für alle die es nicht lesen können sollen. Den SuS sollte klar werden dass dies im modernen Zeitalter mit Kommunikation über das Internet nötig ist, da generell alle Daten die gesendet werden, für alle leserlich sind. Die einzige Möglichkeit Daten geheim zu halten, ist indem man die Daten verschlüsselt bevor sie gesendet werden.

In dieser Lerneinheit werden also nicht nur Verschlüsselungsverfahren vorgestellt, sondern auch Konzepte vermittelt, weshalb wir Verschlüsselung eigentlich brauchen. Da dies ein komplexes Thema ist, und viele Aspekte hat, ist es wichtig für die SuS sich innerhalb von dem Thema zu orientieren, und das gelernte auch zu überprüfen. Dazu dient das Quiz am Ende von jedem Kapitel.

\section*{Quiz}

Das Quiz stellt einfache bis schwierige Fragen zu dem Thema, welche einen Anhaltspunkt geben, wie gut das gelernte verstanden wurde. Die FRagen sollen auch dazu dienen, dass sich die SuS aktiv Gedanken zu dem Thema machen sollen, und somit die Selbstreflexion unterstützen. Zur Selbstüberprüfung dienen die Punkte in einem Quiz, was ein Anhaltspunkt sein soll ein Thema nochmals zu wiederholen und das Quiz dann nochmals durchführen. So könne die SuS zwar aktiv nach den Antworten für das Quiz suchen, dies unterstützt jedoch schon den aktiven Lernprozess und macht den SuS klar was sie gelernt haben (Selbstreflexion).

\section*{Interaktive animierte Lerneinheiten}

In den Lerneinheiten wird das Thema welches die SuS bearbeiten beschrieben. Wenn möglich wird diese Beschreibung durch interaktive und animierte Elemente unterstützt. Die Animationen sollen dazu dienen den Fokus der SuS besser zu halten, und die Konzepte gleich zu visualisieren. Dadurch sollen mehr Verarbeitungskanäle bei den SuS aktiviert werden, wodurch sie sich stärker mit dem Thema auseinandersetzen sollen. Die Animationen sollen das ganze auch ein wenig auflockern, was die Motivation für das Thema steigern soll.

Die Interaktivität ist dadurch gegeben dass die SuS die Inhalte anpassen können, und dadurch auch die Animationen gleich angepasst werden. Damit wird denn SuS ermöglicht einfach kleine Anpassungen zu machen, und zu sehen was dann passiert. Damit sollen die SuS bereits während dem erarbeiten des Themas die dritte bzw. vierte Stufe der Bloom'schen Taxonomie erreichen können. Die SuS sollen das gelernte also anwenden und analysieren können.

\section*{Webseite der Lerneinheit}

Die Lerneinheit ist noch in Entwicklung, kann aber unter der folgenden Adresse aufgerufen werden: \url{https://cedricgeissmann.github.io/sle}

Updates zu der Lerneinheit werden automatisch neu geladen auf der Seite wenn man diese besucht. Zur zeit ist nur ein einfacher Prototyp verfügbar der zeigt wie das Quiz funktioniert und wie die Animationen interaktiv angepasst werden können. Bessere Gestaltung und mehr Inhalt sowie bessere Fragen für das Quiz sind geplant und werden in den nächsten Tagen noch umgesetzt.

\end{document}